\documentclass{beamer}

%
% Choose how your presentation looks.
%
% For more themes, color themes and font themes, see:
% http://deic.uab.es/~iblanes/beamer_gallery/index_by_theme.html
%
\mode<presentation>
{
  \usetheme{default}      % or try Darmstadt, Madrid, Warsaw, ...
  \usecolortheme{default} % or try albatross, beaver, crane, ...
  \usefonttheme{default}  % or try serif, structurebold, ...
  \setbeamertemplate{navigation symbols}{}
  \setbeamertemplate{caption}[numbered]
  \setbeamertemplate{footline}[page number]
  \setbeamercolor{frametitle}{fg=white}
  \setbeamercolor{footline}{fg=black}
} 

\usepackage[english]{babel}
\usepackage[utf8x]{inputenc}
\usepackage{tikz}
\usepackage{listings}
\usepackage{courier}

\xdefinecolor{darkblue}{rgb}{0.1,0.1,0.7}
\xdefinecolor{dianablue}{rgb}{0.18,0.24,0.31}
\definecolor{commentgreen}{rgb}{0,0.6,0}
\definecolor{stringmauve}{rgb}{0.58,0,0.82}

\lstset{ %
  backgroundcolor=\color{white},      % choose the background color
  basicstyle=\ttfamily\small,         % size of fonts used for the code
  breaklines=true,                    % automatic line breaking only at whitespace
  captionpos=b,                       % sets the caption-position to bottom
  commentstyle=\color{commentgreen},  % comment style
  escapeinside={\%*}{*)},             % if you want to add LaTeX within your code
  keywordstyle=\color{blue},          % keyword style
  stringstyle=\color{stringmauve},    % string literal style
  showstringspaces=false,
  showlines=true
}

\lstdefinelanguage{scala}{
  morekeywords={abstract,case,catch,class,def,%
    do,else,extends,false,final,finally,%
    for,if,implicit,import,match,mixin,%
    new,null,object,override,package,%
    private,protected,requires,return,sealed,%
    super,this,throw,trait,true,try,%
    type,val,var,while,with,yield},
  otherkeywords={=>,<-,<\%,<:,>:,\#,@},
  sensitive=true,
  morecomment=[l]{//},
  morecomment=[n]{/*}{*/},
  morestring=[b]",
  morestring=[b]',
  morestring=[b]"""
}

\title[2016-07-25-future-software-psych]{Assessing future software options}
\author{Jim Pivarski}
\institute{Princeton -- DIANA}
\date{July 25, 2016}

\begin{document}

\logo{\pgfputat{\pgfxy(0.11, 8)}{\pgfbox[right,base]{\tikz{\filldraw[fill=dianablue, draw=none] (0 cm, 0 cm) rectangle (50 cm, 1 cm);}}}\pgfputat{\pgfxy(0.11, -0.6)}{\pgfbox[right,base]{\tikz{\filldraw[fill=dianablue, draw=none] (0 cm, 0 cm) rectangle (50 cm, 1 cm);}\includegraphics[height=0.99 cm]{diana-hep-logo.png}\tikz{\filldraw[fill=dianablue, draw=none] (0 cm, 0 cm) rectangle (4.9 cm, 1 cm);}}}}

\begin{frame}
  \titlepage
\end{frame}

\logo{\pgfputat{\pgfxy(0.11, 8)}{\pgfbox[right,base]{\tikz{\filldraw[fill=dianablue, draw=none] (0 cm, 0 cm) rectangle (50 cm, 1 cm);}\includegraphics[height=1 cm]{diana-hep-logo.png}}}}

% Uncomment these lines for an automatically generated outline.
%\begin{frame}{Outline}
%  \tableofcontents
%\end{frame}

\begin{frame}{Terminology}
\vspace{0.25 cm}

\begin{block}{Hallway warning:}
``Even if software X is better, physicists won't use it because it's unfamiliar.''
\begin{itemize}
\item Supported by anecdotal evidence. Can we get systematic?
\end{itemize}
\end{block}

\begin{block}{The Cuisinart problem:}
It slices and dices, but do you want to set it up and clean it?
\begin{itemize}
\item Physicists regularly have to make pragmatic decisions about whether the overhead in using a tool is worthwhile.
\end{itemize}
\end{block}

\begin{block}{Top-down versus bottom-up:}
Collaboration frameworks, filesystems, and cluster infrastructures get decided upon by a few people and used by many.

\vspace{0.25 cm}
End-user analysis tools get evaluated by each user.
\end{block}
\end{frame}

\begin{frame}{Sociology of physicists}
\vspace{0.35 cm}
I'm planning a focus group to study physicists' opinions on
\begin{itemize}
\item relative importance of familiar syntax
\item relative importance of computational speed
\item relative importance of collaborators using same tools
\item relative importance of size of community (in and beyond HEP)
\item willingness to try advanced techniques: analysis in a cloud, functional chains, metaprogramming
\item possibly with Spark/Julia examples to make things concrete
\end{itemize}

\vspace{0.25 cm}
The format is
\begin{itemize}
\item closed/anonymous, but producing a public statement by concensus
\item limited to physicsts actively typing analysis code (mostly students/postdocs); about ten participants
\end{itemize}
\end{frame}

\begin{frame}{This is a starting point}
I've been reading up on focus groups; usually they should be repeated a few times to find out if you've fully sampled the space of opinions or if dominant voices biased one of the groups.

\vspace{0.35 cm}
Moreover, I have no experience in this form of research. Would we be able to attract interest from social scientists who can do this kind of study professionally?

\vspace{0.35 cm}
Focus groups are often the starting point with surveys as a follow-up. (The open-ended focus groups determine what questions should go on the survey and how to phrase them.)

\vspace{0.35 cm}
Might we analyze HEP software users in a routine way with regular, scientific surveys?
\end{frame}

\begin{frame}{Backup: summary of Julia talks}
\vspace{0.25 cm}
\textcolor{darkblue}{(including points not raised)}

\vfill

\begin{itemize}
\item Julia uniquely ``feels like Python'' and ``runs as fast as C.''
\item Has built-in multiprocessing through ssh, like a lightweight PROOF cluster. \textcolor{gray}{(Killer app not mentioned???)}
\item Calls out to C, C++, Python, R, and Java.
\item Has quirks like 1-based indexing and lack of OOP-like methods: {\tt \scriptsize event.track(2).hit(55)} $\to$ {\tt \scriptsize hit(track(event, 2), 55)}
\item Lack of compile-time safety could be bridged by reflection.
\item Demonstrated interface with ROOT, but issues with conflicting LLVM in Cling.
\item Expects major additions before version 1.0 (e.g. threading).
\item Many packages are in early stages of development.
\item Relatively small community outside HEP, compared to Hadoop/Spark, Scikit-Learn, or R.
\end{itemize}
\end{frame}

\end{document}
